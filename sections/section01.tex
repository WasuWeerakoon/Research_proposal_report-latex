\section{Introduction}

\subsection{Background of the Study}
Provide background information about your research topic. Use citations appropriately. 

\textit{Example:}

\textit{Let's assume a research study titled 'A Reinforcement Learning Approach for Phishing Detection.'} \\ \\
With the rapid expansion of internet services, the need for effective cybersecurity has become more critical than ever. As individuals and organizations increasingly rely on digital platforms for communication, transactions, and data storage, the risk of cyber threats has grown proportionally. Among various cyberattacks, social engineering techniques—especially phishing—have emerged as a prevalent and damaging form of attack, capable of deceiving even experienced users into revealing sensitive information.

Phishing attacks are designed to trick users into providing personal credentials such as usernames, passwords, and financial information, typically through fraudulent emails or websites that mimic legitimate ones. These attacks are often difficult to detect due to their dynamic nature and the ever-evolving tactics used by attackers. Traditional phishing detection techniques, which rely heavily on blacklists or rule-based systems, often fail to keep up with zero-day threats or previously unseen attack patterns.

To overcome these limitations, machine learning approaches have been increasingly applied to phishing detection. While supervised learning has shown promise, it depends on large volumes of labeled data and may struggle with adapting to new patterns in phishing behavior. Recent research interest has turned toward reinforcement learning (RL) — a learning paradigm that enables models to improve through interaction with their environment. RL-based systems can potentially learn to detect phishing websites by sequentially exploring and evaluating features, offering adaptability and decision-making in uncertain or changing conditions. However, despite its theoretical potential, the application of RL in phishing detection remains limited and underexplored.

Given the dynamic nature of phishing attacks and the adaptability required to detect them effectively, this study aims to investigate the feasibility and performance of reinforcement learning techniques in phishing website detection. By developing and evaluating an RL-based model, the research seeks to contribute a novel, adaptive solution that addresses the shortcomings of traditional and supervised learning approaches. The findings of this study may offer insights into how intelligent agents can be used to improve real-time threat detection and cybersecurity resilience.

\subsection{Problem Statement}
Clearly state the research problem that your study aims to address.

Para 1: Start by briefly explaining the broader area of your research. Provide context and explain why this domain is important or relevant today. Mention any current trends, challenges, or real-world importance to set the stage. Narrow the focus to a particular issue or gap within the broader area. Clearly state what is lacking, ineffective, or not well understood. Show why this problem matters. Describe what negative outcomes may occur if the issue is not addressed. 

Para 2: Mention what previous researchers or systems have done and explain where they fall short. Identify any areas that have not been fully explored or problems that existing methods do not adequately solve.

Para 3: Summarize the problem in a clear way. Clearly communicate what the issue is, why it is significant, and why it needs to be studied now. This final paragraph becomes your official problem statement. (This paragraph is what you wrote during Lesson 3.)

Use citations where necessary.

\subsection{Aim, Objectives and Questions}
Explain the aim of your research. List the specific objectives of your study. State any hypotheses you plan to test (if applicable). List research questions that your study seeks to answer.

\begin{itemize}
	\item Aim: Aim goes here.
	\item Objectives:
	\begin{itemize}
		\item Objective 1 goes here.
		\item Objective 2 goes here.
		\item Objective 3 goes here
	\end{itemize} system.
	\item Research Questions:
	\begin{itemize}
		\item Question 1 goes here
		\item Question 2 goes here
		\item Question 3 goes here
	\end{itemize}
\end{itemize}

\subsection{Rationale of the Study}
Explain why you are doing this research or why the study is needed.

\subsection{Significance of the Study}
Explain why the results are important. Think from academic and social perspectives. 

\subsection{Scope and Limitations of the Study}
Define the boundaries of your research by outlining the specific aspects or areas it will cover. Additionally, mention any limitations or constraints that may impact the extent or depth of your study.